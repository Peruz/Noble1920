\documentclass[a4paper,12pt]{article}
\usepackage[utf8]{inputenc} % input encdoding = written input text to latex
\usepackage[T1]{fontenc} % encoding = compiled latex to output font
\usepackage{natbib}% support (author-year, numbered, etc.), also improved compatibility with bib managing
\usepackage{microtype} % microtypography is the art of enhancing the appearance of a document while exhibiting the minimum degree of visual obtrubsion
\usepackage{newpxtext,newpxmath} % vectorial palatino-like font with support for math, siunitx, mchem, etc. It is also darker than standard latex font
\usepackage[left=2cm,right=2cm,top=3cm,bottom=3cm,headheight=15pt]{geometry} % adjust layout - borders
\usepackage{emptypage} % avoid headers and footers on white pages.
\usepackage[parfill]{parskip} % pararaph layout (white vertical space, last paragraph line with minimum space, etc.)
\usepackage{makecell}
\usepackage{multirow} % macros for tabular env. etc. (multirow cells, ...)
\usepackage{tabularx,ragged2e} % extention of tabular with column X, which automatically adjusts width etc
\usepackage{float} % improved management (H, ...) for floating objects
\usepackage{bm} % bold in math mode (e.g., vector fields)
\usepackage{graphicx} % extention of the graphics package, including key-value interface for the includegraphics command 
\usepackage{caption}
\usepackage{subcaption} % for formatting captions and subcaptions
\usepackage{booktabs} % for professional looking tables
\usepackage[version=4]{mhchem} % chemical formulas
\usepackage{amsfonts}
\usepackage{pdflscape}
\usepackage{amssymb} % extended set of fonts and symbols for mathematics
\usepackage{amsmath} % improve structure and output of documents that contain mathematics
\usepackage{mathtools} % based on amsmath, it improves looking and provide new capabilities and options (numbering, format, etc.)
\usepackage{pdfpages}
\usepackage{textcomp} % extended set of fonts and symbols
\usepackage[toc,page]{appendix} % allows appendices
\usepackage{epigraph}
\setlength\epigraphwidth{0.7\textwidth}
\setlength\epigraphrule{0pt} % allow epighaphs and provide various handles to tweak the appearance
\usepackage{listings} % include code
\usepackage{siunitx} % standard input of units and numbers (SI, si, num, etc.)
\sisetup{separate-uncertainty, multi-part-units=single, detect-family, detect-weight, range-units=single}
\usepackage[compact]{titlesec} % managing titles, headers, etc. Compact reduces spaces above and below the titles
\usepackage{fancyhdr}
\usepackage{hyperref}
\usepackage{xcolor}
\hypersetup{colorlinks, linkcolor=[rgb]{0.1,0.2,0.7}, citecolor=[RGB]{120,0,0}, urlcolor=[rgb]{0.1,0.2,0.7}}
\title{}
\author{Luca Peruzzo}
\date{}
\begin{document}
\setcounter{secnumdepth}{0}
\newcommand{\myseparator}{\noindent\makebox[\linewidth]{\resizebox{0.5\linewidth}{1pt}{$\bullet$}}\bigskip}
\newcommand{\subwidth}{0.48}
\graphicspath{{./../}{./../figures/}{./figures/}}
%\maketitle
%\begin{figure}[H]
%\centering
%\begin{subfigure}{\textwidth}
%\includegraphics[width=\textwidth]{}
%\end{subfigure}
%\end{figure}

\section{Sensors at Noble}

Sensors and data logger are from  METER ENVIRONMENT group, check their website for quick doubts.

The data logger is a EM60G.
There are 6 sensors installed: 4 5TE (moisture, temperature, and EC) and 2 TEROS 21 (water potential and temperature).
The two types of sensors were installed in two close wells (1-2 feet apart) that are about the center of the ERT line (~ m 15 along the line, ~ 1 m away from the line).
The 4 5TE sensors were installed at a depth of 20, 40, 60, and 80 cm.
The 2 TEROS 21 were installed at a depth of 20 and 40 cm.

\subsection{Download the data}

I think we have two options.

The easiest is trying to reach the ERT laptop from the data logger, without moving the laptop so that we don't have to play with the ERT cables.
We installed the needed software on the laptop, and simply connecting the two would make it (we can access with teamviewer).
I'm not sure what is the actual distance between laptop and data logger but I would say within 2 m, based on the installation details above.
Thus, I think it may be worth trying either with the cable we sent you or we a new one (maybe adding an USB extension).

Although secondary, the lid of the data logger cannot be closed while the micro-USB cable is connected.
Even if we use this solution, we wont be able to have continuous access with teamviewer (as for the ERT).

If reaching the laptop is harder than expected, here are the steps to install the software.
Once in the field, we could again help with teamviewer if installed and needed.

\subsubsection{Install the software}
\begin{enumerate}
\item Download and install "ZENTRA Utility installer" from the METER Environment website.
\begin{itemize}
\item It should be possible to open this link \url{https://www.metergroup.com/environment/articles/buy-browse-meter-legacy-data-loggers/}
\item This is the page for the data logger. Under EM60 open "Manual and Software" and and click "ZENTRA Utility installer" (PC or MAC).
\end{itemize}
\item Open Zentra Utility on desktop (Big White Z) and check it seems ok.
\end{enumerate}

\subsubsection{In the field}
\begin{itemize}
\item Open Zentra.
\item Connect the data logger to the laptop with a USB - microUSB cable (we shipped one with the sensors if needed),
\item Select the right comport and click 'Connect'.
\item Click 'Download' and have the options to download the preferable range of days (start from ~ March 15).
\end{itemize}

\end{document}
